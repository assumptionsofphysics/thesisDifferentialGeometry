\documentclass{book}


\begin{document}

\tableofcontents

\chapter{Mathematical representation of physical objects}

\section{Topology and experimental distinguishability}
To define physical objects we should be able to experimentally tell them apart. This gives us a topology.

\section{Manifolds and continuous quantities}
We tell physical object apart by identifying some measurable properties. In particular, we are interested in properties that are continuous quantities. This gives us manifolds.

\subsection{Coordinate transformations}
Physics is equally valid regardless of coordinate system. Therefore, if we can describe a physical phenomenon in one frame of reference, we should be able to convert that description into another frame of reference. 
\subsection{Parametrizations of curves}
Continuing the idea of validity in multiple views, we may describe the same line, curve, or surface with different, but generally equally valid functions
\chapter{Mathematical representation of infinitesimal objects}

\section{Differentiable manifolds and densities}
We are interested in studying distribution on manifolds. Talk about them as a measure. The limit on infinitesimal areas is a density and requires coordinates to be differentiable.

\section{Vectors and infinitesimal displacements}
At any point on a manifold, we can calculate the tangent plane. An infinitesimal displacement along this plane (a directional derivative) is a vector. 
\section{Covectors and linear functions of displacements}
We can define functions that convert infinitesimal displacements to a scalar value. These functions may be 
\section{Tensors and coordinate transformations}
Some quantities can undergo one-to-one transformations between coordinate systems. In general, such quantities are tensors. 
\section{K-forms and wedge product}
In more abstract or complicated coordinate systems, it becomes useful to define geometry on infinitesimal displacements. 
\chapter{Geometry and ???}

\section{Riemannian geometry}
Geometry of infinitesimal displacements is made necessary by a constantly changing metric tensor. 
\subsection{Metric tensor}
A tool for calculating distances/displacements along a coordinate system
\subsection{Orthogonal basis}

\section{Symplectic geometry and state spaces}
The density of points in our system remains constant under change in position. 

\subsection{Symplectic form and areas}

\end{document}