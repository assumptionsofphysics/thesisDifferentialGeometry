\documentclass{book}


\begin{document}

\tableofcontents

\chapter{Mathematical representation of physical objects}

\section{Topology and experimental distinguishability}
\textbf{Consider whether this is necessary for the thesis. If I can't get it straight for myself, I won't be able to relate it to others effectively}
\textit{To define physical objects we should be able to experimentally tell them apart. This gives us a topology.}

\section{Manifolds and continuous quantities}
\textit{We tell physical object apart by identifying some measurable properties. In particular, we are interested in properties that are continuous quantities. This gives us manifolds.}

\subsection{Physical objects}

Say we have some object that we want to identify, and we have an idea in mind of all the possible cases we may identify it as. The ultimate goal is to be able to experimentally determine the range of cases that the object belongs to. 
 
Now imagine you get up one morning, and you want to know what the weather will be like that day in Ann Arbor, MI. You check the forecast on your phone, and see that it will be a balmy 33$^{\circ}$F. Because the report is made with only so many significant figures, the actual temperature may be anywhere between 32.1 and 33.1. So, what you actually have is a range of possible cases that you've established that the real temperature is within. Furthermore, there are many other possible ranges that your case didn't lie within. For instance, the range could have been from 0.1 to 1.1$^{\circ}$F, or 100.1 to 101.1, etc. And still further, we can see that the ranges of theoretically possible temperatures together cover between just above absolute zero and whatever maximum temperature may exist (however unlikely most of the temperatures may be outside of some very unfortunate scenario). Besides temperature, we could just so describe air pressure, humidity, wind speed, chance of snow, your whereabouts, whatever. 

\subsection{Quantities}
$\textit{In quantifying the properties that we want to study, we want to be able to compare two objects with the same "sort" of property, and maybe we also want to identify multiple parameters at the same time.}$

When we have a 
\subsection{Coordinate transformations}
$\textit{Physics is equally valid regardless of coordinate system. Therefore, if we can describe a physical phenomenon in one frame of reference, we should be able to convert that description into another frame of reference. }$
 


\subsection{Parametrizations of curves}
Continuing the idea of validity in multiple views, we may describe the same line, curve, or surface with different, but generally equally valid functions
\textbf{TODO: } Define k-surfaces and the boundary operator.
\textbf{TO DO:} Given some parametrization in $S_{k}$, what would be the parametrization in $S_{k-1}$ when a boundary operator is applied?

\chapter{Mathematical representation of infinitesimal objects}

\section{Differentiable manifolds and densities}
We are interested in studying distribution on manifolds. Talk about them as a measure. The limit on infinitesimal areas is a density and requires coordinates to be differentiable.

\section{Vectors and infinitesimal displacements}
At any point on a manifold, we can calculate the tangent plane. An infinitesimal displacement along this plane (a directional derivative) is a vector. 
\section{Covectors and linear functions of displacements}
We can define functions that convert infinitesimal displacements to a scalar value. These functions may be 
\section{Tensors and coordinate transformations}
Some quantities can undergo one-to-one transformations between coordinate systems. In general, such quantities are tensors. 
\section{K-forms and wedge product}
In more abstract or complicated coordinate systems, it becomes useful to define geometry on infinitesimal displacements. 
\chapter{Geometry and ???}

\section{Riemannian geometry}
In order to give a mathematically rigorous definition of lengths of vectors and the angle between them, we use an inner product. Vector spaces with an inner product are Riemannian. 
\subsection{Metric tensor}
The inner product itself: feed in two vectors, and the outcome is a scalar representing the lengths of the vectors and the angle between them. 
\subsection{Orthogonal basis}

\section{Symplectic geometry and state spaces}
We want to be able to represent our state configurations. Symplectic geometry arises when trying to describe their areas

\subsection{Symplectic form and areas}
Just as the metric tensor lets us define length and angles, the symplectic form lets us define areas. 

\end{document}