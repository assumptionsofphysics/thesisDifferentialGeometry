\documentclass{book}


\begin{document}

\tableofcontents

\chapter{Mathematical representation of physical objects}

\section{Topology and experimental distinguishability}
To define physical objects we should be able to experimentally tell them apart. This gives us a topology.

\section{Manifolds and continuous quantities}
We tell physical object apart by identifying some measurable properties. In particular, we are interested in properties that are continuous quantities. This gives us manifolds.

\subsection{Coordinate transformations}

\subsection{Parametrizations of curves}

\chapter{Mathematical representation of infinitesimal objects}

\section{Differentiable manifolds and densities}
We are interested in studying distribution on manifolds. Talk about them as a measure. The limit on infinitesimal areas is a density and requires coordinates to be differentiable.

\section{Vectors and infinitesimal displacements}

\section{Covectors and linear functions of displacements}

\section{Tensors and coordinate transformations}

\section{K-forms and wedge product}

\chapter{Geometry and ???}
\section{Riemannian geometry}
\subsection{Metric tensor}
\subsection{Orthogonal basis}
\section{Symplectic geometry and state spaces}
\subsection{Symplectic form and areas}

\end{document}