\documentclass{book}
\usepackage{mathtools}

\usepackage{amsthm}
\usepackage{amsmath}
\usepackage{amssymb}

\newtheorem{defn}[equation]{Definition}
%\newtheorem{corollary}[defn]{Corollary} Figure out how to set this up


\begin{document}

\tableofcontents

\chapter{Mathematical representation of physical objects}



\section{Manifolds and continuous quantities}
\emph{We tell physical object apart by identifying some measurable properties. In particular, we are interested in properties that are continuous quantities. This gives us manifolds.}

\subsection{Physical objects and quantities}

- \emph{There is a physical object that we want to identify}

Say we're looking at weather. We'll want to know the temperature, wind speed, humidity, whereabouts, etc.

- \emph{We can assign a real number value to it, this gives us quantities}

Fahrenheit, mph, percent humidity, location 




\begin{defn}
	A \textbf{quantity} is a function $q : U \to \mathbb{R}$ that assigns a measurable value to a physical object.
\end{defn}


- \emph{Having a set of quantities gives us a coordinate system}

\begin{defn}
	A \textbf{coordinate system} $\vec{q}$ is a collection of $n$ quantities $q^i : U \to \mathbb{R}$ such that there is a one-to-one relationship between the physical objects in $U$ and the values of the quantities in $\mathbb{R}^n$.
\end{defn}


- \emph{Coordinate systems can be transformed between}

\begin{defn}
	Given two coordinate systems  $\vec{q} : U \to \mathbb{R}^n$ and $\vec{p} : V \to \mathbb{R}^n$ such that $U \cap V \neq \emptyset$, we call a \textbf{coordinate transformation} the function $f = \vec{p} \circ (\vec{q})^{-1} : \mathbb{R}^n \to \mathbb{R}^n$.
\end{defn}




- \emph{Now that we have sets that can be mapped to the real numbers, we can define manifolds}

For the weather example, a manifold would all possible weather conditions (including things that can't necessarily be quantified, like the color of a sunset), with a subsection whereon is defined a weather report, measuring temperature, wind speed, etc. in different places. 
 
\begin{defn}
	A \textbf{manifold} is a set of physical objects $X$ such that for any $x \in X$ there exists a $U \subset X$ that contains $x$ and upon which a coordinate system $\vec{q}$ is defined.
\end{defn}




\subsection{Sub-manifolds and k-surfaces}

- \emph{Within a manifold can lie a manifold of a smaller dimension, with a mapping between them}
If in the original weather report we were worrying about temperature, humidity, and wind speed, perhaps now we're only worrying about temperature and humidity. The latter would be a sub-manifold of the former. 

\begin{defn}
	Given some manifold X of dimension n, a \textbf{sub-manifold} of dimension m is a manifold $Y \subset X$ such that $m \leq n$, with an injective coordinate transformation $f: \mathbb{R}^m \to \mathbb{R}^n$. 
\end{defn}
The ignored dimensions must be held fixed, otherwise the original manifold is recovered. 

\begin{defn}
	For some manifold X of dimension n, $S_k$ is the set of all possible \textbf{k-surfaces}, or sub-manifolds of dimension $k \leq n$. 
\end{defn}

- \emph{A manifold of dimension n can have an edge of dimension n-1. The boundary itself will not have a boundary.}
The boundary of a weather report would be 

\begin{defn}
	Given k-surface $\sigma \subset S_k \subset X$, $\forall x \in \sigma$, $\exists$ a neighborhood $U \subset X$ containing $x$ homeomorphic to an open set in $\mathbb{R}^k$. The \textbf{boundary} of $\sigma$, denoted $\partial\sigma$, consists of points that are not homeomorphic to an open set in $\mathbb{R}^k$. 
\end{defn}

%\begin{corollary}
%	Given $\sigma \subset S_k$, $\partial\sigma \subset S_{k-1}$, and $\partial\partial\sigma = \emptyset$. 
%	\end{corollary} Should be a way to set this up so it numbers as a subsection of a definition

	





\chapter{Mathematical representation of infinitesimal objects}



\section{Differentiable manifolds and densities}
We are interested in studying distribution on manifolds. Talk about them as a measure. The limit on infinitesimal areas is a density and requires coordinates to be differentiable.


\begin{defn}
	Given some manifold $X$ of dimension $n$, with overlapping subsets $U$ and $V$ with defined coordinate systems $\vec{q}: U \to \mathbb{R}^n$ and $\vec{p}: V \to \mathbb{R}^n$, if the coordinate transformation $f = \vec{q} \circ \vec{p}^{-1}$ is smooth, then $X$ is a \textbf{differentiable manifold}. 
\end{defn}

\begin{defn}
	Given $U \subset X$, where X is a manifold, a $\textbf{measure}$ is a function $f : U \to \mathbb{R}^n$ for non-negative elements of $\mathbb{R}^n$. 
\end{defn}

\begin{defn}
	Given measure $f$ defined on X, the \textbf{density} of X is the limit of ...?
\end{defn}



\section{Vectors and infinitesimal displacements}
At any point on a manifold, we can calculate the tangent plane. An infinitesimal displacement along this plane (a directional derivative) is a vector. 


\section{Covectors and linear functions of displacements}
We can define functions that convert infinitesimal displacements to a scalar value. 

\subsection{Linear Functionals of K-surfaces}


\begin{defn}
	A \textbf{linear function of k-surfaces} is a function $F : S_k \to \mathbb{R}$ such that for $\sigma_1$, $\sigma_2 \in S_k$, if $\sigma_1 \cap \sigma_2 \subseteq \partial\sigma_1 \cup \partial\sigma_2$, then $F(\sigma_1\cup\sigma_2) = F(\sigma_1) + F(\sigma_2)$. 
\end{defn}


Given $F : S_k \to \mathbb{R}$, we can then define $G : S_{k+1} \to \mathbb{R}$ such that $G(\sigma) = F(\partial\sigma)$, and $H : S_{k+2} \to \mathbb{R}$ such that $H(\sigma) = G(\partial\sigma) = F(\partial\partial\sigma)$. 

\begin{defn}
	A linear function may be defined \textbf{infinitesimally}, such that $F : S_k \to \mathbb{R} = \int_{\sigma} f(d\sigma)$. Further, for $G : S_{k+1} \to \mathbb{R}$, $G(\sigma) = \int_{\sigma} g(d\sigma) = F(\partial\sigma) = \int_{\partial\sigma}f(d\partial\sigma)$. 
\end{defn}


\section{Tensors and coordinate transformations}
Some quantities can undergo one-to-one transformations between coordinate systems. In general, such quantities are tensors. 

\section{K-forms and wedge product}
In more abstract or complicated coordinate systems, it becomes useful to define geometry on infinitesimal displacements. 

\begin{defn}
	A \textbf{differential} such as dx, dy, dz describes an infinitesimal displacement along a coordinate axis. 
\end{defn}

\begin{defn}
	Given vectors $u, v \in U$, the \textbf{wedge product} $u \wedge v: V \times V \to...$ 
\end{defn}
\begin{defn}
	A \textbf{differential k-form} is made by wedging together k different differentials, which can be used to describe the length, area, volume, etc. of an appropriately dimensioned space. 
\end{defn}


\chapter{Geometry and ???}

\section{Riemannian geometry}
In order to give a mathematically rigorous definition of lengths of vectors and the angle between them, we use an inner product. Vector spaces with an inner product are Riemannian.

\begin{defn}
	Given two vectors $u,v \in V$, the \textbf{inner product} $<u,v> : V \times V \to \mathbb{R}$ is defined as $<u,v> = |u||v|cos(\theta)$, where $\theta$ is the angle between u and v. 
\end{defn}
\begin{defn}
	Given two vectors $u,v \in V$, the vector space V whereon is defined an \textbf{inner product} $<u,v> : V \times V \to \mathbb{R}$ is \textbf{Riemannian}.
	\end{defn}


\subsection{Metric tensor}
The inner product itself: feed in two vectors, and the outcome is a scalar representing the lengths of the vectors and the angle between them. 
\subsection{Orthogonal basis}

\section{Symplectic geometry and state spaces}
We want to be able to represent our state configurations. Symplectic geometry arises when trying to describe their areas

\subsection{Symplectic form and areas}
Just as the metric tensor lets us define length and angles, the symplectic form lets us define areas. 

\end{document}