\documentclass{book}
\usepackage{mathtools}

\usepackage{amsthm}
\usepackage{amsmath}
\usepackage{amssymb}

\newtheorem{defn}[equation]{Definition}
\newtheorem{coro}[equation]{Corollary}
\newtheorem{prf}[equation]{Proof}


\begin{document}

\tableofcontents

\chapter{Mathematical representation of physical objects}



\section{Manifolds and continuous quantities}
\emph{We tell physical object apart by identifying some measurable properties. In particular, we are interested in properties that are continuous quantities. This gives us manifolds.}

\subsection{Physical objects and quantities}

- \emph{There is a physical object that we want to identify}

Say we're looking at weather. We'll want to know the temperature, wind speed, humidity, whereabouts, etc.

- \emph{We can say it lives in some subset of possibilities, and can give real number values corresponding to that subset. This gives us quantities}

Fahrenheit, mph, percent humidity, location.
On some set of all possible weather configurations, we look at a specific subset of possibilities. This is our weather report.  






\begin{defn}
	A \textbf{quantity} is an invertible function $q : U \to \mathbb{R}$ that assigns a measurable value to a physical object.
\end{defn}


- \emph{Having a set of linearly independent quantities sufficient to distinguish one object from another gives us a coordinate system}




\begin{defn}
	A \textbf{coordinate system} $[q]$ is a collection of $n$ quantities $q^i : U \to \mathbb{R}$ such that there is a one-to-one relationship between the physical objects in $U$ and the values of the quantities in $\mathbb{R}^n$.
\end{defn}


- \emph{\textbf{Overlapping} coordinate systems can be transformed between}



\begin{defn}
	Given two coordinate systems  $[q] : U \to \mathbb{R}^n$ and $[p] : V \to \mathbb{R}^n$ such that $U \cap V \neq \emptyset$, we call a \textbf{coordinate transformation} the function $f = [p] \circ ([q])^{-1} : \mathbb{R}^n \to \mathbb{R}^n$.
\end{defn}




- \emph{Now that we have sets that can be mapped to the real numbers, we can define manifolds}

For the weather example, a manifold would be all possible weather conditions (including things that can't necessarily be quantified, like the color of a sunset), with a subsection whereon is defined a weather report, measuring temperature, wind speed, etc. in different places. If we're looking to parametrize the entire earth, then our manifold cannot be covered with a single subset, because the earth is roughly a sphere. We will need multiple subsets to cover all of the earth, and all possible weather conditions. 


 
\begin{defn}
	A \textbf{manifold} is a set of physical objects $X$ such that for any $x \in X$ there exists a $U \subset X$ that contains $x$ and upon which a coordinate system $[q]$ is defined.
\end{defn}

\begin{defn}
	For a given manifold X, a collection of subsets of X that completely cover X are called the \textbf{atlases} of X. 
\end{defn}



\subsection{Sub-manifolds and k-surfaces}

- \emph{Within a manifold can lie a manifold of equal or smaller dimension, with a mapping between them}

If in the original weather report we were worrying about temperature, humidity, and wind speed, we could instead hold one of the three values fixed (say, humidity), and vary the other two. The latter manifold, with humidity held constant, would be a sub-manifold of the former, where all three variables are allowed to change.  



\begin{defn}
	Consider a manifold X of dimension n, and a manifold Y of dimension m, such that $m \leq n$. Y is a \textbf{submanifold} of X if:
	
	1. Any point on Y is also a point on X; and,
	
	2. there are points on X that are not on Y. 
\end{defn}

\begin{coro}
	Suppose we have a manifold X of dimension n, with a submanifold Y of dimension m, such that $m \leq n$. Recalling the definition of submanifolds, this means that any point that lies in Y also lies in X, but the inverse is not true. Therefore, the coordinate transformation from X to Y will be onto, while the coordinate transformation from Y to X will be one-to-one. 
\end{coro}


\emph{What are the possible submanifolds of a given manifold?}


\begin{defn}
	For some manifold X of dimension n, $S_k$ is the set of all possible \textbf{k-surfaces}, or sub-manifolds of dimension $k \leq n$. 
\end{defn}

- \emph{A manifold of dimension n can have a boundary of dimension n-1. The boundary itself will not have a boundary.}

The boundary of a weather report would be readings restricted to a specific climate. 


\begin{defn}
	Given k-surface $\sigma \in S_k$, if for some point $x \in \sigma$ you can construct a neighborhood around x completely within $\sigma$, then x is an \textbf{interior point} of $\sigma$. 
\end{defn}

\begin{defn}
	Given k-surface $\sigma \in S_k$, if some point $x \in \sigma$ is not an interior point (i.e., a neighborhood around x entirely within $S_k$ cannot be made), then $x$ is a \textbf{boundary point. }
\end{defn}

\begin{defn}
	Given k-surface $\sigma \in S_k$, $\partial\sigma \in S_{k-1}$ is the set of all boundary points of $\sigma$, or, the \textbf{boundary} of $\sigma$. 
\end{defn}


\begin{coro}
	Given $\sigma \subset S_k$, $\partial\sigma \subset S_{k-1}$, and $\partial\partial\sigma = \emptyset$. 
\end{coro}

TODO: parametrization of the boundary (\emph{consider a cube, rather than a ball})

\subsection{Linear Functionals of K-surfaces}


\begin{defn}
	A \textbf{linear function of k-surfaces} is a function $F : S_k \to \mathbb{R}$ such that for $\sigma_1$, $\sigma_2 \in S_k$, if $\sigma_1 \cap \sigma_2 \subseteq \partial\sigma_1 \cup \partial\sigma_2$, then $F(\sigma_1\cup\sigma_2) = F(\sigma_1) + F(\sigma_2)$. 
\end{defn}


Given $F : S_k \to \mathbb{R}$, we can then define $G : S_{k+1} \to \mathbb{R}$ such that $G(\sigma) = F(\partial\sigma)$, and $H : S_{k+2} \to \mathbb{R}$ such that $H(\sigma) = G(\partial\sigma) = F(\partial\partial\sigma)$. 

\begin{prf}
	We want to show that a linear functional applied to the empty set gives zero. 
	
	Let $F : S_k \to \mathbb{R}$ be a linear function. Consider the empty set $\emptyset$. 
	Trivially, $\emptyset \cap \emptyset \subseteq \partial\emptyset \cup \partial\emptyset$, and $\emptyset = \emptyset\cup\emptyset$. 
	
	So, $F(\emptyset) = F(\emptyset\cup\emptyset) = F(\emptyset) + F(\emptyset) \implies F(\emptyset) = 2F(\emptyset) \implies F(\emptyset) = 0$
	$\qedsymbol$
\end{prf}

TODO: notation for F $\to$ G (\emph{I've asked a couple people about this. General consensus is that the notation is unfortunate, but it is a clear representation of what's happening})

\begin{prf}
	We want to show that a linear functional applied to a surface whereon the boundary operator was applied twice will always give zero.
	
	Let $F : S_k \to \mathbb{R}, G : S_{k+1} \to \mathbb{R}, and H : S_{k+2} \to \mathbb{R}$ be linear functions, and let $\sigma$ be an element of $S_k$. $H(\sigma) = G(\partial\sigma) = F(\partial\partial\sigma) = F(\emptyset) = 0$. 
	
	So, $H(\sigma) = 0$ $\forall$ $\sigma$, and H is the zero function. 
	$\qedsymbol$
\end{prf}






\chapter{Mathematical representation of infinitesimal objects}



\section{Differentiable manifolds and densities}
We are interested in studying distribution on manifolds. Talk about them as a measure. The limit on infinitesimal areas is a density and requires coordinates to be differentiable.


\begin{defn}
	Given some manifold $X$ of dimension $n$, with overlapping subsets $U$ and $V$ with defined coordinate systems $[q]: U \to \mathbb{R}^n$ and $[p]: V \to \mathbb{R}^n$, if the coordinate transformation $f = [q] \circ [p]^{-1}$ is smooth, then $X$ is a \textbf{differentiable manifold}. 
\end{defn}

\begin{defn}
	Given $U \subset X$, where X is a manifold, a $\textbf{measure}$ is a function $f : U \to \mathbb{R}^n$ for non-negative elements of $\mathbb{R}^n$. 
\end{defn}

\begin{defn}
	Given measure $f$ defined on X, the \textbf{density} of X is the limit of ...?
\end{defn}



\section{Vectors and infinitesimal displacements}
At any point on a manifold, we can calculate the tangent plane. An infinitesimal displacement along this plane (a directional derivative) is a vector. 


\section{Covectors and linear functions of displacements}
We can define functions that convert infinitesimal displacements to a scalar value. 

TODO: Think about the notation for differential of a boundary. It's kind of ugly. The previous statement of others confirming that it's clear, however, still applies. 

\begin{defn}
	A linear function may be defined \textbf{infinitesimally}, such that $F : S_k \to \mathbb{R} = \int_{\sigma} f(d\sigma)$. Further, for $G : S_{k+1} \to \mathbb{R}$, $G(\sigma) = \int_{\sigma} g(d\sigma) = F(\partial\sigma) = \int_{\partial\sigma}f(d\partial\sigma)$. 
\end{defn}


\section{Tensors and coordinate transformations}
Some quantities can undergo one-to-one transformations between coordinate systems. In general, such quantities are tensors. 

\emph{Use some variant of the definition used in d'Inverno}

\section{K-forms and wedge product}
In more abstract or complicated coordinate systems, it becomes useful to define geometry on infinitesimal displacements. 

\begin{defn}
	A \textbf{differential} such as dx, dy, dz describes an infinitesimal displacement along a coordinate axis. 
\end{defn}

\begin{defn}
	Given vectors $u, v \in U$, the \textbf{wedge product} $u \wedge v: V \times V \to V$, and describes the area of a parallelogram with sides u and v. 
\end{defn}

- \emph{We want to be able to describe displacements and functions along them independent of coordinate system. This gives us differential forms}

TODO: \emph{\textbf{Definitely fix up this definition of differential forms}}

\begin{defn}
	A \textbf{differential k-form} is a linear function of contravariant vectors along covariant differential elements used to describe the length, area, volume, etc. of an appropriately dimensioned space 
\end{defn}


\chapter{Geometry and ???}

\section{Riemannian geometry}
In order to give a mathematically rigorous definition of lengths of vectors and the angle between them, we use an inner product. Vector spaces with an inner product are Riemannian.

\begin{defn}
	Given two vectors $u,v \in V$, the \textbf{inner product} $<u,v> : V \times V \to \mathbb{R}$ is defined as $<u,v> = |u||v|cos(\theta)$, where $\theta$ is the angle between u and v. 
\end{defn}
\begin{defn}
	Given two vectors $u,v \in V$, the vector space V whereon is defined an \textbf{inner product} $<u,v> : V \times V \to \mathbb{R}$ is \textbf{Riemannian}.
	\end{defn}


\subsection{Metric tensor}
A more generalized inner product: feed in two vectors, and the outcome is a scalar representing the lengths of the vectors and the angle between them. 
\subsection{Orthogonal basis}

\section{Symplectic geometry and state spaces}
We want to be able to represent our state configurations. Symplectic geometry arises when trying to describe their areas

\subsection{Symplectic form and areas}
Just as the metric tensor lets us define length and angles, the symplectic form lets us define areas. 

\end{document}